\section{Hacker's Delight}\label{sec:book}
\subsection*{The Book}
Henry Warren's book ``\emph{Hacker's Delight}'' is described by himself as ``a collection of small programming tricks'' \cite{Warren:2012:HD:2462741} (p. \texttt{xv}).
In fact, the table of contents reads like an encyclopedia of formulas for bit level manipulation:
From detecting, counting and changing specific single bits to formulas that fulfill a mathematical goal (and interpret the bitword as a number, not a vector), everything imaginable seems to be part of this collection.

\subsection*{This Paper}
Inspired by and named after ``\emph{Hacker's Delight}'',
this paper complements it with methods and explanations that have been left out by the author (intentionally):

\begin{quote}
``The presentation is informal. Proofs are given only when the algorithm is not obvious, and sometimes not even then.''

\hfill -- \emph{Hacker's Delight}, page \texttt{xv} \cite{Warren:2012:HD:2462741}
\end{quote}

\noindent
Instead of simply explaining given formulas,
this essay assembles them from a few elementary operations in a stepwise manner.
Besides a fundamental understanding of the covered topics,
the aim is to provide the reader with methods that can be adapted and thereby used to create new formulas.
