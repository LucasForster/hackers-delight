\section{Hacker's Delight}\label{sec:book}
\epigraph{
``There is nothing here about hacking
in the sense of breaking into computers.''
}{
\emph{``Hacker's Delight''} Website
\cite{Warren:HD:Website}
}


\subsection*{The Book}
Henry Warren's book ``\emph{Hacker's Delight}''
is described by himself as
``a collection of small programming tricks''
\cite{Warren:2012:HD:2462741} (p. \texttt{xv}).
In fact, the table of contents reads like
an encyclopedia of formulas for bit level manipulation:

It starts out with detecting, counting and changing
specific bits in a bitword.
From there on, everything imaginable seems to be part of this collection.
This includes bit-level manipulations that result
in a mathematical meaning (interpreting the bitword as a number).

The bypassing of common algebraic steps
to achieve performance gains is the typical example
of what should be considered a ``hack''.


\subsection*{This Paper}
Inspired by and named after ``\emph{Hacker's Delight}'',
this paper complements it with explanations
that have been left out by the author (intentionally):

\begin{quote}
``The presentation is informal.
Proofs are given only when the algorithm is not obvious,
and sometimes not even then.''
\par\hfill -- \emph{Hacker's Delight},
page \texttt{xv} \cite{Warren:2012:HD:2462741}
\end{quote}

Instead of simply explaining given formulas,
this essay assembles them from a few elementary operations
in a stepwise manner.
Besides a fundamental understanding of the covered topics,
the aim is to provide the reader with methods
that can be adapted and thereby used to create new formulas.


\subsection*{The Book's Title}
As addressed in the epigraph, ``hacker'' refers to
someone enjoying clever ways of coding,
not necessarily being a professional
or creating useful work.
The ``delight'' is developed when faced with the elegance of some solutions.
