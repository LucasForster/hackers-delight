\section{Introduction}\label{sec:introduction}
\epigraph{
``It is amazing what can be done
with just [...] some bitwise operations.''
}{
\emph{Hacker's Delight}, page \texttt{xiii}
\cite{Warren:2012:HD:2462741}
}


\subsection*{Crazy Ideas in Data Structures and Algorithms}
The seminar this contribution is part of covers
ideas in computer science with a certain ``craziness''.
This might not obviously apply to this paper:

The underlying data structure is a bitword
-- the simplest interpretation of binary values possible.
Furthermore, all formulas are composed of simple logical operators,
without any complex (e.g. recursive) nesting involved.

Where the ``craziness'' starts however,
is in the rejection of abstract data types (like boolean-arrays).
Addressing single bits isn't directly possible.
Instead ``hacks'' will be used that involve decoupling
operations (e.g. increment) from their mathematical meaning
(see \autoref{sec:rightmost}).
Additionally, instruction-level parallelism will allow for
``crazy'' fast computation: everything fitting in a bitword
is processed by the CPU in a single time step.


\subsection*{Bitwords}
The common term \emph{bitword} (or \emph{bitvector}).
It is simply defined as a sequence of bits with fixed length
that is indexed (starting with $0$) from right to left
(e.g. in \autoref{table:bitword} ``$\uparrow$'' refers to $x_4$).
There isn't any further interpretation (e.g. as a numerical value) to it.
This holds true when applying operations with such a meaning
(e.g. incrementing).

A good way to think of a bitvector is a series of
\emph{on} (\lstinline$1$) and \emph{off} (\lstinline$0$) values.
Terms like ``rightmost'' and ``trailing''
are used in their conventional meaning (see \autoref{table:bitword})
-- others will be introduced when needed.

\begin{table}[H]
\centering
\begin{tabular}{r}
$\underset{\text{\hfill rightmost \lstinline$1$}}
{\text{\lstinline$x = 1100 1011 001$}
{\underset{\uparrow}{\text{\lstinline$1$}}}}
\underbrace{\text{\lstinline$0000$}}_{\text{trailing \lstinline$0$s}}$\\
\end{tabular}
\caption{Bitword}
\label{table:bitword}
\end{table}


\subsection*{Logical Operators}
The mathematical logical operators serve as a basis
for all bit manipulation.
They are interpreted in the well-known way (see \autoref{table:logic}).

\begin{table}[H]
\centering
\begin{tabular}{c|c||c|c|c}
  \lstinline$x$ & \lstinline$y$
& \lstinline$NOT$: $\lnot$ \lstinline$x$
& \lstinline$OR$: \lstinline$x$ $\lor$ \lstinline$y$
& \lstinline$AND$: \lstinline$x$ $\land$ \lstinline$y$\\
\hline\hline
  \lstinline$0$ & \lstinline$0$
& \lstinline$1$ & \lstinline$0$ & \lstinline$0$\\
\hline
  \lstinline$0$ & \lstinline$1$
& \lstinline$1$ & \lstinline$1$ & \lstinline$0$\\
\hline
  \lstinline$1$ & \lstinline$0$
& \lstinline$0$ & \lstinline$1$ & \lstinline$0$\\
\hline
  \lstinline$1$ & \lstinline$1$
& \lstinline$0$ & \lstinline$1$ & \lstinline$1$\\
\end{tabular}
\caption{Basic boolean operators}
\label{table:logic}
\end{table}

The set $\{\lnot, \lor, \land\}$ presented in \autoref{table:logic}
is considered a \emph{complete set of logical operators},
which means that any logical formula can be expressed
only using elements from this set.
There is an important distinction to be made
on how to apply a logical operator to a bitword
(see \autoref{table:operators} for \emph{C}):

\begin{description}
\item[byte-level:] This common interpretation
defines the whole variable (at least 1 byte in size)
as a single boolean value being false iff all bits are \lstinline$0$.

\item[bitwise:] This lesser used version
is made for bit-level manipulation.
As the name suggests every bit of the argument is treated independently.

\end{description}

\begin{table}[H]
\centering
\begin{tabular}{c|c|c|c}
function & mathematical & \emph{C}: byte-level & \emph{C}: bitwise\\
\hline
\lstinline$NOT$ & $\lnot$ & \lstinline$!$ & \lstinline$~$\\
\lstinline$OR$ & $\lor$ & \lstinline$||$ & \lstinline$|$\\
\lstinline$AND$ & $\land$ & \lstinline$&&$ & \lstinline$&$\\
\end{tabular}
\caption{Byte-level and bitwise operators in \emph{C}}
\label{table:operators}
\end{table}


\subsection*{Mathematical Operators}
In addition to manipulating bits independently,
operators that interpret sequence as a whole a required.
Increment (\lstinline$INC$, \lstinline$(x+1)$ in \emph{C})
and decrement (\lstinline$DEC$, \lstinline$(x-1)$ in \emph{C})
do so by treating the bitwords in a mathematical sense.
The assigning notations \lstinline$x++$ and \lstinline$x--$
are ignored to avoid side effects.

When executing these operation, \emph{carry} and \emph{borrow}
refer to the values that get passed to adjacent bits
(see \autoref{sec:rightmost}).
