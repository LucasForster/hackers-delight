\section{Introduction}\label{sec:introduction}
\epigraph{
``It is amazing what can be done
with just [...] some bitwise operations.''
}{
\emph{``Hacker's Delight''}, page \texttt{xiii}
\cite{Warren:2012:HD:2462741}
}


\subsection*{Crazy Ideas in Data Structures and Algorithms}
This essay is part of a seminar covering
ideas in computer science that feature a certain ``craziness''.
The content that will be presented here
might however not obviously fit into that description:

The underlying data structure is a bitword
-- the simplest interpretation of binary values possible.
Furthermore, all formulas are composed of simple logical operators,
without any complex (e.g. recursive) nesting involved.

None the less, the general principle of rejecting
common (high-level) abstract data types like boolean-arrays
for bitwords turns out to be a pretty crazy idea:
Single bits aren't directly accessible (requires copying and shifting)
and operations like increment will be used for
purposes differentiating a lot from their intended mathematical meaning.
