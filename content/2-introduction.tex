% TODO rename subsections to directly describe their content
\section{Introduction}\label{sec:introduction}
\epigraph{
``It is amazing what can be done
with just [...] some bitwise operations.''
}{
\emph{``Hacker's Delight''}, page \texttt{xiii}
\cite{Warren:2012:HD:2462741}
}


% TODO crazy fast instruction level parallelism
\subsection*{Crazy Ideas in Data Structures and Algorithms}
This essay is part of a seminar covering
ideas in computer science that feature a certain ``craziness''.
The content that will be presented here
might however not obviously fit into that description:

The underlying data structure is a bitword
-- the simplest interpretation of binary values possible.
Furthermore, all formulas are composed of simple logical operators,
without any complex (e.g. recursive) nesting involved.

None the less, the general principle of rejecting
common (high-level) abstract data types like boolean-arrays
for bitwords turns out to be a pretty crazy idea:
Single bits aren't directly accessible (requires copying and shifting)
and operations like increment will be used for
purposes differentiating a lot from their intended mathematical meaning.

\subsection*{Technical Terms}
The most common term that will be used (and has been)
is a ``bitword'' or ``bitvector''.
It is simply defined as a sequence of bits with fixed length
that is indexed starting with $0$ from right to left
(so in \autoref{fig:bitword} ``$\uparrow$'' refers to $x_3$).

However, it is important to remember that
there isn't any advanced interpretation (like a numerical value) to it
unless otherwise noted.
This still holds true while executing operations
which are designed for such interpretation (e.g. incrementing).
Therefore a good way to think of a bitvector
is a series of ``on'' (\lstinline$1$) and ``off'' (\lstinline$0$) values.

% TODO using aligned, add trailing/rightmost into figure
\begin{figure}[h]
\[
x = \lstinline$1100 1011 001 $
\underset{\uparrow}{\lstinline$1$}
\underbrace{\lstinline$000$}
\]
\caption{
A bitword \lstinline$x$ (length $16$)
with marked rightmost \lstinline$1$ and trailing \lstinline$0$s
}
\label{fig:bitword}
\end{figure}

Terms like ``rightmost'' and ``trailing''
keep their conventional meaning (see \autoref{fig:bitword})
-- the other ones will be introduced when required.
