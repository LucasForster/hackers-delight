% TODO: rename section and subsections
\section{Manipulating Rightmost Bits}\label{sec:rightmost}
\epigraph{
``The fact that the carry chain allows a single bit
to affect all the bits to its left
makes addition a peculiarly powerful data manipulation operation''
}{
\emph{``Hacker's Delight''}, page \texttt{xiv}
\cite{Warren:2012:HD:2462741}
}


\subsection*{Motivation}
As mentioned in \autoref{sec:book},
the goal is to complement ``Hacker's Delight''.
More concretely, the book starts out in chapter 2 with a list of formulas.
While it is possible to understand why they work as they do,
it is a completely different task to create these by yourself
-- this needs a systematic approach.


\subsection*{Idea}
Exchanging the known mathematical meaning of any known operator
for a more \emph{visual approach} will allow to develop
a basic understanding of bit manipulation.
Consider the \lstinline$INC$ operator as an example:

Similar to performing addition by hand,
start at the least significant bit.
Add \lstinline$1$ to it:
If it was \lstinline$0$, it becomes \lstinline$1$
and the incrementation is done.
If however the value was already \lstinline$1$, it becomes \lstinline$0$
and the overflow is carried over to the next bit,
where the process is repeated.
So starting from the right, every bit gets inverted
until the first \lstinline$0$-bit is found.

\begin{table}[H]
\centering
\begin{tabular}{rl}
mathematical:
& $x \mapsto (x+1) \text{ mod } 2^n$ (unsigned)\\
~\\
step by step:
& \hspace{4ex}$\text{\lstinline$0110 011$}
    \underset{\uparrow}{\text{\lstinline$1$}}$\\
& 1) add \lstinline$1$ to the least significant bit\\
& \hspace{4ex}$\text{\lstinline$0110 01$}
    \underset{\uparrow}{\text{\lstinline$1$}}\text{\lstinline$0$}$\\
& 2) while carrying, repeat with next higher bit\\
& \hspace{4ex}$\dots$\\
& \hspace{4ex}$\text{\lstinline$0110\ $}
    \underset{\uparrow}{\text{\lstinline$1$}}\text{\lstinline$000$}$\\
~\\
visual:
& \fbox{invert all bits up to the rightmost \lstinline$0$}\\
& \hspace{4ex}$\text{\lstinline$0110\ $}
    \underbrace{\text{\lstinline$0111$}}_{\text{invert}}$\\
& \hspace{4ex}\lstinline$0110 1000$
\end{tabular}
\caption{Visual interpretation of the \lstinline$INC$ operator}
\label{table:idea-inc}
\end{table}


\subsection*{Visual Approach}
The last section gave a detailed insight
on how the \lstinline$INC$ operator is visually interpretable.
The counterpart \lstinline$DEC$ follows an analog pattern,
therefore an overview should suffice to introduce it.
But first, in order to have all operators presented in the following schema,
the \lstinline$INC$ operator will be repeated in \autoref{table:inc}.

\begin{table}[H]
\centering
\begin{tabular}{ll}
\lstinline$INC$: & example:\\
1) add \lstinline$1$ to the least significant bit
& $\text{\lstinline$0110 011$}\underset{\uparrow}{\text{\lstinline$1$}}$\\
2) while carrying, repeat with next higher bit
& $\text{\lstinline$0110 01$}
    \underset{\uparrow}{\text{\lstinline$1$}}\text{\lstinline$0$}$\\
& \lstinline$0110 1000$\\
\fbox{invert all bits up to the rightmost \lstinline$0$}
& $\text{\lstinline$0110\ $}
    \underbrace{\text{\lstinline$0111$}}_{\text{invert}}$\\
& \lstinline$0110 1000$
\end{tabular}
\caption{The \lstinline$INC$ operator}
\label{table:inc}
\end{table}

\begin{table}[H]
\centering
\begin{tabular}{ll}
\lstinline$DEC$: & example:\\
1) remove \lstinline$1$ to the least significant bit
& $\text{\lstinline$0110 100$}\underset{\uparrow}{\text{\lstinline$0$}}$\\
2) while carrying, repeat with next higher bit
& $\text{\lstinline$0110 10$}
    \underset{\uparrow}{\text{\lstinline$0$}}\text{\lstinline$1$}$\\
& \lstinline$0110 0111$\\
\fbox{invert all bits up to the rightmost \lstinline$1$}
& $\text{\lstinline$0110\ $}
    \underbrace{\text{\lstinline$1000$}}_{\text{invert}}$\\
& \lstinline$0110 0111$
\end{tabular}
\caption{The \lstinline$DEC$ operator}
\label{table:dec}
\end{table}

In \autoref{sec:introduction} logical operators where introduced
along the mathematical ones that have just been covered.
The bitwise version will be of interest,
with the \lstinline$NOT$ operator being trivial
regarding the visual interpretation.

Left with \lstinline$OR$ as well as \lstinline$AND$,
the approach will be to minimise the effort of determining the result
by only looking at certain bits of the first argument
and ``modifying'' the second argument to become the result.

\begin{table}[H]
\centering
\begin{tabular}{ll}
\lstinline$OR$: & example:\\
& \lstinline$x = 0011$\\
& \lstinline$y = 0101$\\
for each resulting bit, determine
$\text{\lstinline$r$}_i = \text{\lstinline$x$}_i \lor \text{\lstinline$y$}_i$
& \lstinline$r = 0111$\\
~\\
& $\text{\lstinline$y = 0$}\underset{\downarrow}{\text{\lstinline$1$}}
    \text{\lstinline$0$}\underset{\downarrow}{\text{\lstinline$1$}}$\\
\fbox{for each \lstinline$1$ in \lstinline$y$,
turn on the corresponding bit in \lstinline$x$}
& \lstinline$x = 0011$\\
& \lstinline$r = 0111$\\
\end{tabular}
\caption{The \lstinline$OR$ operator}
\label{table:or}
\end{table}

\begin{table}[H]
\centering
\begin{tabular}{ll}
\lstinline$AND$: & example:\\
& \lstinline$x = 1100$\\
& \lstinline$y = 1010$\\
for each resulting bit, determine
$\text{\lstinline$r$}_i = \text{\lstinline$x$}_i \land \text{\lstinline$y$}_i$
& \lstinline$r = 1000$\\
~\\
& $\text{\lstinline$y = 1$}\underset{\downarrow}{\text{\lstinline$0$}}
    \text{\lstinline$1$}\underset{\downarrow}{\text{\lstinline$0$}}$\\
\fbox{for each \lstinline$0$ in \lstinline$y$,
turn off the corresponding bit in \lstinline$x$}
& \lstinline$x = 1100$\\
& \lstinline$r = 1000$\\
\end{tabular}
\caption{The \lstinline$AND$ operator}
\label{table:and}
\end{table}

Notice that there was made a specific choice
in \autoref{table:or} and \autoref{table:and}
to let the second argument \lstinline$y$
manipulate the first argument \lstinline$x$
instead of the other way round.
This will enhance readability in formulas
presented in \autoref{sec:combining}.

In summary the visual interpretations of operators
(including the skipped \lstinline$NOT$ operator)
presented in \autoref{table:visual} are obtained.

\begin{table}[H]
\centering
\begin{tabular}{cc|l}
Operator & & Visual Interpretation\\
\hline
\lstinline$NOT$ & \lstinline$~x$
    & invert all bits\\
\lstinline$INC$ & \lstinline$x+1$
    & invert all bits up to the rightmost \lstinline$0$\\
\lstinline$DEC$ & \lstinline$x-1$
    & invert all bits up to the rightmost \lstinline$1$\\
\lstinline$OR$  & \lstinline$x|y$
    & for each \lstinline$1$ in \lstinline$y$,
        turn on the corresponding bit in \lstinline$x$\\
\lstinline$AND$ & \lstinline$x&y$
    & for each \lstinline$0$ in \lstinline$y$,
        turn off the corresponding bit in \lstinline$x$\\
\end{tabular}
\caption{Summary of visual interpretations of selected operators}
\label{table:visual}
\end{table}
