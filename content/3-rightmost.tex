\section{A Visual Approach to Operations}\label{sec:rightmost}
\epigraph{
``The fact that the carry chain allows a single bit
to affect all the bits to its left
makes addition a peculiarly powerful data manipulation operation''
}{
\emph{Hacker's Delight}, page \texttt{xiv}
\cite{Warren:2012:HD:2462741}
}

As mentioned in \autoref{sec:book},
the goal is to complement \emph{Hacker's Delight}:
chapter 2 of the book simply lists formulas.
While it is possible to understand why they work as they do,
it is a completely different task to create these on your own
-- this needs a systematic approach.

In order to develop the interpretations described by Henry Warren,
a \emph{visual approach} to operations has to be obtained first.
Consider the \lstinline$INC$ operator as an example:
Similar to performing addition by hand,
the value \lstinline$1$ gets added onto the least significant bit.
In the case of performing \lstinline$1+1$,
the overflow gets carried over to the next bit.

This repeats until the first \lstinline$0$ is reached.
So starting from the right, every bit gets inverted
until the first \lstinline$0$-bit is found.
This latter description is considered
the visual interpretation of \lstinline$INC$.
\lstinline$DEC$ follows an analog pattern
to the presented \lstinline$INC$ operator.
This allows for the visual interpretations
shown in \autoref{table:inc} and \autoref{table:dec}.

\begin{table}[H]
\centering
\begin{tabular}{r|ccc}
\lstinline$x$ & left of rightmost \lstinline$0$
    & rightmost \lstinline$0$ & trailings \lstinline$1$s\\
\hline
\lstinline$(x+1)$ & unchanged & \lstinline$1$ & \lstinline$0...0$\\
& \multicolumn{3}{c}{
    \fbox{invert all bits up to the rightmost \lstinline$0$}}
\end{tabular}
\caption{\lstinline$INC$ (visual interpretation)}
\label{table:inc}
\end{table}

\begin{table}[H]
\centering
\begin{tabular}{r|ccc}
\lstinline$x$ & left of rightmost \lstinline$1$
    & rightmost \lstinline$1$ & trailings \lstinline$0$s\\
\hline
\lstinline$(x-1)$ & unchanged & \lstinline$0$ & \lstinline$1...1$\\
& \multicolumn{3}{c}{
    \fbox{invert all bits up to the rightmost \lstinline$1$}}
\end{tabular}
\caption{\lstinline$DEC$ (visual interpretation)}
\label{table:dec}
\end{table}

After the now covered mathematical operators from
\autoref{sec:introduction},
the bitwise logical operators are of interest.
\lstinline$NOT$ is the trivial one to visually interpret,
because it simply inverts all bits of a given bitword.

Regarding \lstinline$OR$ and \lstinline$AND$,
the approach is to reduce the original definition of calculating
\lstinline$r$$_i$\lstinline$ = x$$_i \lor$\lstinline$y$$_i$ and
\lstinline$r$$_i$\lstinline$ = x$$_i \land$\lstinline$y$$_i$ respectively
for each bit $i$.

Instead, one might think of the second argument
manipulating the first one to become the result:
Set \lstinline$r$ to \lstinline$x$, then
for every \lstinline$1$ in \lstinline$y$
set the corresponding bit in \lstinline$r$ to \lstinline$1$.
Set to \lstinline$0$ for every \lstinline$0$ in \lstinline$y$
when executing \lstinline$AND$.
This is summarised in \autoref{table:or} and \autoref{table:and}.

\begin{table}[H]
\centering
\begin{tabular}{r|cc}
\lstinline$y$ & bits that are \lstinline$0$
    & bits that are \lstinline$1$\\
\hline
\lstinline$(x|y)$ & \lstinline$x$ (unchanged) & \lstinline$1$\\
& \multicolumn{2}{c}{\fbox{for each \lstinline$1$ in \lstinline$y$,
    set to \lstinline$1$ in \lstinline$x$}}
\end{tabular}
\caption{\lstinline$OR$ (visual interpretation)}
\label{table:or}
\end{table}

\begin{table}[H]
\centering
\begin{tabular}{r|cc}
\lstinline$y$ & bits that are \lstinline$0$
    & bits that are \lstinline$1$\\
\hline
\lstinline$(x&y)$ & \lstinline$0$ & \lstinline$x$ (unchanged)\\
& \multicolumn{2}{c}{\fbox{for each \lstinline$0$ in \lstinline$y$,
    set to \lstinline$0$ in \lstinline$x$}}
\end{tabular}
\caption{\lstinline$AND$ (visual interpretation)}
\label{table:and}
\end{table}

Notice that there was made a specific choice
in \autoref{table:or} and \autoref{table:and}
to let the second argument \lstinline$y$
manipulate the first argument \lstinline$x$
instead of the other way round.
This will enhance readability in formulas
presented in \autoref{sec:combining}.
