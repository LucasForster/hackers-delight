\section{Manipulating Rightmost Bits}\label{sec:rightmost}
\epigraph{
``The fact that the carry chain allows a single bit
to affect all the bits to its left
makes addition a peculiarly powerful data manipulation operation''
}{
\emph{``Hacker's Delight''}, page \texttt{xiv}
\cite{Warren:2012:HD:2462741}
}


\subsection*{Motivation}
As mentioned in \autoref{sec:book},
the goal is to complement ``Hacker's Delight''.
More concretely, the book starts out in chapter 2 with a list of formulas.
While it is possible to understand why they work as they do,
it is a completely different task to create these by yourself
-- this needs a systematic approach.


\subsection*{Idea}
Exchanging the known mathematical meaning of any known operator
for a more \emph{visual approach} will allow to develop
a basic understanding of bit manipulation.
Consider the \lstinline$INC$ operator as an example:

Similar to performing addition by hand,
start at the least significant bit.
Add \lstinline$1$ to it:
If it was \lstinline$0$, it becomes \lstinline$1$
and the incrementation is done.
If however the value was already \lstinline$1$, it becomes \lstinline$0$
and the overflow is carried over to the next bit,
where the process is repeated.
So starting from the right, every bit gets inverted
until the first \lstinline$0$-bit is found.

\begin{table}[h]
\centering
\begin{tabular}{rl}
mathematical:
& $x \mapsto (x+1) \text{ mod } 2^n$ (unsigned)\\
~\\
step by step:
& \hspace{4ex}$\text{\lstinline$0110 011$}
    \underset{\uparrow}{\text{\lstinline$1$}}$\\
& 1) add \lstinline$1$ to the least significant bit\\
& \hspace{4ex}$\text{\lstinline$0110 01$}
    \underset{\uparrow}{\text{\lstinline$1$}}\text{\lstinline$0$}$\\
& 2) while carrying, repeat with next higher bit\\
& \hspace{4ex}$\dots$\\
& \hspace{4ex}$\text{\lstinline$0110\ $}
    \underset{\uparrow}{\text{\lstinline$1$}}\text{\lstinline$000$}$\\
~\\
visual:
& \fbox{invert all bits up to the rightmost \lstinline$0$}\\
& \hspace{4ex}$\text{\lstinline$0110\ $}
    \underbrace{\text{\lstinline$0111$}}_{\text{invert}}$\\
& \hspace{4ex}\lstinline$0110 1000$
\end{tabular}
\caption{Visual interpretation of the \lstinline$INC$ operator}
\label{table:idea}
\end{table}
