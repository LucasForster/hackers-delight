\section{Combining Operators}\label{sec:combining}
% TODO: Add epigraph (if fitting quote exists)
% TODO: Move zero-testing out of the current document part

\subsection*{Motivation}
In \autoref{sec:introduction} two types of operators where introduced
with their visual interpretation presented in \autoref{sec:rightmost}:
Operators that manipulate bits independently
(\lstinline$OR$, \lstinline$AND$)
as well as operators that chain through multiple bits of the argument
(\lstinline$INC$, \lstinline$DEC$).

Combining pairs from these two sets will allow
to create more complex formulas.
\lstinline$NOT$ will be left aside for the moment.


\subsection*{Idea}
Working a single bitwords \lstinline$x$, we can
manipulate \lstinline$x$ with \lstinline$INC$, \lstinline$DEC$
and can combine the result with the original argument
to add or remove certain bits of \lstinline$x$
using \lstinline$OR$, \lstinline$AND$.

Taking a look at the visual interpretation
of \lstinline$DEC$ as well as \lstinline$AND$
in \autoref{table:dec} and \autoref{table:and},
the combination of these two in the form \lstinline$x AND (DEC x)$
inverts all bits up to the rightmost \lstinline$1$
and turns off every bit in \lstinline$x$ that hasn't been inverted:

Inverting means that one argument of \lstinline$AND$ is always \lstinline$0$
while the other bits are untouched and therefore keep their value.
Since all bits up to the rightmost \lstinline$1$ are \lstinline$0$ anyways,
the resulting formula effectively only turns off the rightmost \lstinline$1$
in \lstinline$x$.


\subsection*{Combining Two Operators}
The already developed formula
consisting of \lstinline$DEC$ and \lstinline$AND$
is presented in \autoref{table:dec-and}.
Notice that there is additionally a ``zero-testing'' field:
Since every power of two is a single \lstinline$1$-bit in a bitword,
the check tests if the input \lstinline$x$ is a power of $2$ (or $0$).

\begin{table}[H]
\centering
\begin{tabular}{ll}
\lstinline$DEC$:
& invert all bits up to the rightmost \lstinline$1$\\
\lstinline$AND$:
& for each \lstinline$0$ in \lstinline$DEC x$,
    turn off the corresponding bit in \lstinline$x$\\
\lstinline$x AND (DEC x)$:
& \fbox{turn off the rightmost \lstinline$1$ in \lstinline$x$}\\
\lstinline$!(x&(x-1))$
& zero-testing: ``is \lstinline$x$ $0$ or a power of $2$?''\\
\end{tabular}
\caption{\lstinline$x AND (DEC x)$}
\label{table:dec-and}
\end{table}

By switching the two operators for their counterparts
\lstinline$INC$ and \lstinline$OR$,
the meaning of the formula gets switched as well, see \autoref{table:inc-or}.
Notice that there isn't any purpose in zero-checking the result however.

\begin{table}[H]
\centering
\begin{tabular}{ll}
\lstinline$INC$:
& invert all bits up to the rightmost \lstinline$0$\\
\lstinline$OR$:
& for each \lstinline$1$ in \lstinline$INC x$,
    turn on the corresponding bit in \lstinline$x$\\
\lstinline$x OR (INC x)$:
& \fbox{turn on the rightmost \lstinline$0$ in \lstinline$x$}\\
\end{tabular}
\caption{\lstinline$x OR (INC x)$}
\label{table:inc-or}
\end{table}

Using the defined sets of operators, two other combinations are possible,
which handle multiple (trailing) bits instead of a single (rightmost) one.
As already seen with when using \lstinline$AND$, zero checks are useful.
In the case of \lstinline$INC$ and \lstinline$AND$,
it tests if the value was of the form $0^*1^*$
-- in mathematical terms $2^m-1$ with $m \in [0,n]$
(for a bitword of length $n$).

\begin{table}[H]
\centering
\begin{tabular}{ll}
\lstinline$INC$:
& invert all bits up to the rightmost \lstinline$0$\\
\lstinline$AND$:
& for each \lstinline$0$ in \lstinline$INC x$,
    turn off the corresponding bit in \lstinline$x$\\
\lstinline$x AND (INC x)$:
& \fbox{turn off the trailing \lstinline$1$ in \lstinline$x$}\\
\lstinline$!(x&(x+1))$
& zero-testing: is $x = 0^*1^*$ i.e. $x = 2^m-1, m \in [0,n]$.
\end{tabular}
\caption{\lstinline$x AND (INC x)$}
\label{table:inc-and}
\end{table}

Again switching the two operators
for their counterparts \lstinline$DEC$ and \lstinline$OR$
leads to the meaning of the formula getting switched as well,
see \autoref{table:dec-or}.

\begin{table}[H]
\centering
\begin{tabular}{ll}
\lstinline$DEC$:
& invert all bits up to the rightmost \lstinline$1$\\
\lstinline$OR$:
& for each \lstinline$1$ in \lstinline$DEC x$,
    turn on the corresponding bit in \lstinline$x$\\
\lstinline$x OR (DEC x)$:
& \fbox{turn on the trailing \lstinline$0$ in \lstinline$x$}\\
\end{tabular}
\caption{\lstinline$x OR (DEC x)$             }
\label{table:dec-or}
\end{table}

% TODO: add preliminary result (kept out in regard to document length)

\subsection*{Combining Three Operators}

The operator \lstinline$NOT$ has been kept out for a simple reason:
It's combination with another operator would have produced
familiar functions like \lstinline$NOR$ and \lstinline$NAND$
or at least trivial ones like inverting
everything left of the rightmost \lstinline$0$ or \lstinline$1$ respectively.

With the four above introduced formulas however, new possibilities arise:
% TODO add WHY for 1/2
\begin{enumerate}
\item invert the full formula:\\
    \lstinline$~(x&(x-1))$, \lstinline$~(x|(x+1))$,
    \lstinline$~(x&(x+1))$, \lstinline$~(x|(x-1))$\\
    this swaps on/off up to the rightmost \lstinline$0$/\lstinline$1$
    and inverts the left section
\item invert the incremented/decrement argument:\\
    \lstinline$x&(~(x-1))$, \lstinline$x|(~(x+1))$,
    \lstinline$x&(~(x+1))$, \lstinline$x|(~(x-1))$\\
    this turns on/off the section
    left from the rightmost \lstinline$0$/\lstinline$1$
\item invert the original argument:\\
    \lstinline$(~x)&(x-1)$, \lstinline$(~x)|(x+1)$,
    \lstinline$(~x)&(x+1)$, \lstinline$(~x)&(x-1)$\\
\end{enumerate}

While the first two options deliver foreseeable results,
the inverting of the unchanged argument is the most complex
and promising option.

\begin{table}[H]
\centering
\begin{tabular}{r|ccc}
\lstinline$x$ & left of rightmost \lstinline$0$
    & rightmost \lstinline$0$ & trailings \lstinline$1$s\\
\hline
\lstinline$(~x)$ & inverted & \lstinline$1$ & \lstinline$0...0$\\
\lstinline$(x+1)$ & unchanged & \lstinline$1$ & \lstinline$0...0$\\
\lstinline$(~x)&(x+1)$ & \lstinline$0...0$
    & \lstinline$1$ & \lstinline$0...0$\\
& \multicolumn{3}{c}{
    \fbox{a single \lstinline$1$ at the rightmost \lstinline$0$}}
\end{tabular}
\caption{\lstinline$(NOT x) AND (INC x)$}
\label{table:not-inc-and}
\end{table}

\begin{table}[H]
\centering
\begin{tabular}{r|ccc}
\lstinline$x$ & left of rightmost \lstinline$1$
    & rightmost \lstinline$1$ & trailings \lstinline$0$s\\
\hline
\lstinline$(~x)$ & inverted & \lstinline$1$ & \lstinline$0...0$\\
\lstinline$(x-1)$ & unchanged & \lstinline$0$ & \lstinline$1...1$\\
\lstinline$(~x)|(x-1)$ & \lstinline$1...1$
    & \lstinline$0$ & \lstinline$1...1$\\
& \multicolumn{3}{c}{
    \fbox{a single \lstinline$0$ at the rightmost \lstinline$1$}}\\
\end{tabular}
\caption{\lstinline$(NOT x) OR (DEC x)$}
\label{table:not-dec-or}
\end{table}

\begin{table}[H]
\centering
\begin{tabular}{r|ccc}
\lstinline$x$ & left of rightmost \lstinline$1$
    & rightmost \lstinline$1$ & trailings \lstinline$0$s\\
\hline
\lstinline$(~x)$ & inverted & \lstinline$1$ & \lstinline$1...1$\\
\lstinline$(x-1)$ & unchanged & \lstinline$0$ & \lstinline$1...1$\\
\lstinline$(~x)&(x-1)$ & \lstinline$0...0$
    & \lstinline$0$ & \lstinline$1...1$\\
& \multicolumn{3}{c}{
    \fbox{\lstinline$1$s at trailing \lstinline$0$s}}\\
\end{tabular}
\caption{\lstinline$(NOT x) AND (DEC x)$}
\label{table:not-dec-and}
\end{table}

\begin{table}[H]
\centering
\begin{tabular}{r|ccc}
\lstinline$x$ & left of rightmost \lstinline$0$
    & rightmost \lstinline$0$ & trailings \lstinline$1$s\\
\hline
\lstinline$(~x)$ & inverted & \lstinline$1$ & \lstinline$1...1$\\
\lstinline$(x+1)$ & unchanged & \lstinline$1$ & \lstinline$0...0$\\
\lstinline$(~x)|(x+1)$ & \lstinline$1...1$
    & \lstinline$1$ & \lstinline$0...0$\\
& \multicolumn{3}{c}{
    \fbox{\lstinline$0$s at trailing \lstinline$1$s}}\\
\end{tabular}
\caption{\lstinline$(NOT x) OR (INC x)$}
\label{table:not-inc-or}
\end{table}

Additionally, every formula can be inverted in total.
This is especially useful when the result is ``a single \lstinline$0$''
like for \lstinline$(NOT x) OR (DEC x)$ (see \autoref{table:not-dec-or}).
The result then has the same scheme as \lstinline$(NOT x) AND (INC x)$
(see \autoref{table:not-inc-and}).
