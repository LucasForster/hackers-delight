\section{Combining Operators}\label{sec:combining}
% TODO: Add epigraph (if fitting quote exists)
% TODO: Move zero-testing out of the current document part

\subsection*{Motivation}
In \autoref{sec:introduction} two types of operators where introduced
with their visual interpretation presented in \autoref{sec:rightmost}:
Operators that manipulate bits independently
(\lstinline$OR$, \lstinline$AND$)
as well as operators that chain through multiple bits of the argument
(\lstinline$INC$, \lstinline$DEC$).

Combining pairs from these two sets will allow
to create more complex formulas.
\lstinline$NOT$ will be left aside for the moment.


\subsection*{Idea}
Working a single bitwords \lstinline$x$, we can
manipulate \lstinline$x$ with \lstinline$INC$, \lstinline$DEC$
and can combine the result with the original argument
to add or remove certain bits of \lstinline$x$
using \lstinline$OR$, \lstinline$AND$.

Taking a look at the visual interpretation
of \lstinline$DEC$ as well as \lstinline$AND$ from \autoref{table:visual},
the combination of these two in the form \lstinline$x AND (DEC x)$
inverts all bits up to the rightmost \lstinline$1$
and turns off every bit in \lstinline$x$ that hasn't been inverted:

Inverting means that one argument of \lstinline$AND$ is always \lstinline$0$
while the other bits are untouched and therefore keep their value.
Since all bits up to the rightmost \lstinline$1$ are \lstinline$0$ anyways,
the resulting formula effectively only turns off the rightmost \lstinline$1$
in \lstinline$x$.
