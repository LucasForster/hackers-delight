\section{Combining Operators}\label{sec:combining}
% TODO: Add epigraph (if fitting quote exists)
% TODO: Move zero-testing out of the current document part

Two types of operators have been introduced
in \autoref{sec:introduction}:

\begin{itemize}
\item Operators that manipulate bits independently in parallel:
    \lstinline$OR$, \lstinline$AND$
\item Operators that chain through multiple bits of the argument:
    \lstinline$INC$, \lstinline$DEC$
\end{itemize}

Combining pairs from these two sets will allow
to create more complex formulas.
\lstinline$NOT$ (part of the first group) is left out,
since it would simply invert the result
and thereby not contributing to interesting results.
It will be useful as a third operator later on.


\subsection*{Combining two operators}
Working a single bitwords \lstinline$x$, we can
manipulate \lstinline$x$ with \lstinline$INC$, \lstinline$DEC$
and can combine the result with the original argument
to add or remove certain bits of \lstinline$x$
using \lstinline$OR$, \lstinline$AND$.

\lstinline$x AND (DEC x)$ for example
inverts all bits up to the rightmost \lstinline$1$
and then turns off every bit in \lstinline$x$ that hasn't been inverted.
This means every bit up to the rightmost \lstinline$1$ has been turned off.
This is equivalent to turning off the rightmost \lstinline$1$
(on the right are only trailing \lstinline$0$s anyway).

Other combinations follow a similar pattern as demonstrated in
\autoref{table:dec-and}, \autoref{table:inc-or},
\autoref{table:inc-and} and \autoref{table:dec-or}.

\begin{table}[H]
\centering
\begin{tabular}{r|ccc}
\lstinline$x$ & left of rightmost \lstinline$1$
    & rightmost \lstinline$1$ & trailing \lstinline$0$s\\
\hline
\lstinline$(x-1)$ & unchanged & \lstinline$0$ & \lstinline$1...1$\\
\lstinline$x&(x-1)$ & unchanged
    & \lstinline$0$ & unchanged\\
& \multicolumn{3}{c}{
    \fbox{turn off the rightmost \lstinline$1$}}
\end{tabular}
\caption{\lstinline$x AND (DEC x)$}
\label{table:dec-and}
\end{table}

By switching the two operators for their counterparts
\lstinline$INC$ and \lstinline$OR$,
the meaning of the formula gets switched as well.

\begin{table}[H]
\centering
\begin{tabular}{r|ccc}
\lstinline$x$ & left of rightmost \lstinline$0$
    & rightmost \lstinline$0$ & trailing \lstinline$1$s\\
\hline
\lstinline$(x+1)$ & unchanged & \lstinline$1$ & \lstinline$0...0$\\
\lstinline$x|(x+1)$ & unchanged
    & \lstinline$1$ & unchanged\\
& \multicolumn{3}{c}{
    \fbox{turn on the rightmost \lstinline$0$}}
\end{tabular}
\caption{\lstinline$x OR (INC x)$}
\label{table:inc-or}
\end{table}

Using the defined sets of operators, two other combinations are possible,
which handle multiple (trailing) bits instead of a single (rightmost) one.

\begin{table}[H]
\centering
\begin{tabular}{r|ccc}
\lstinline$x$ & left of rightmost \lstinline$0$
    & rightmost \lstinline$0$ & trailing \lstinline$1$s\\
\hline
\lstinline$(x+1)$ & unchanged & \lstinline$1$ & \lstinline$0...0$\\
\lstinline$x&(x+1)$ & unchanged
    & unchanged & \lstinline$0...0$\\
& \multicolumn{3}{c}{
    \fbox{turn off trailing \lstinline$0$s}}
\end{tabular}
\caption{\lstinline$x AND (INC x)$}
\label{table:inc-and}
\end{table}

Again switching the two operators
for their counterparts \lstinline$DEC$ and \lstinline$OR$
leads to the meaning of the formula getting switched as well.

\begin{table}[H]
\centering
\begin{tabular}{r|ccc}
\lstinline$x$ & left of rightmost \lstinline$1$
    & rightmost \lstinline$1$ & trailing \lstinline$0$s\\
\hline
\lstinline$(x-1)$ & unchanged & \lstinline$0$ & \lstinline$1...1$\\
\lstinline$x|(x-1)$ & unchanged
    & unchanged & \lstinline$1...1$\\
& \multicolumn{3}{c}{
    \fbox{turn on trailing \lstinline$0$s}}
\end{tabular}
\caption{\lstinline$x OR (DEC x)$}
\label{table:dec-or}
\end{table}

\subsubsection*{Zero-Testing}
The two formulas build with \lstinline$AND$ turn off bits.
Therefore their result can be zero-tested (in a meaningful way):
\begin{itemize}
\item \lstinline$x AND (DEC x)$:\\
    \lstinline$!(x&(x-1))$$\equiv$``is \lstinline$x$ $0$ or a power of $2$''
\item \lstinline$x AND (INC x)$:\\
    \lstinline$!(x&(x+1))$$\equiv$``is \lstinline$x$ $2^m - 1, m \in [0,n]$''
    (for a bitword of length $n$)
\end{itemize}


\subsection*{Combining Three Operators}

\lstinline$NOT$ has been left out at the beginning.
With the four above introduced formulas however, new possibilities arise.
There are three options to use the negation on them:
% TODO add WHY for 1/2
\begin{enumerate}
\item invert the full formula:\\
    \lstinline$~(x&(x-1))$, \lstinline$~(x|(x+1))$,
    \lstinline$~(x&(x+1))$, \lstinline$~(x|(x-1))$\\
    this just inverts the original result,
    meaning that:\\
    up to the rightmost \lstinline$0$/\lstinline$1$
    the turning on/off gets swapped\\
    and additionally the left part is inverted
\item invert the incremented/decrement argument:\\
    \lstinline$x&(~(x-1))$, \lstinline$x|(~(x+1))$,
    \lstinline$x&(~(x+1))$, \lstinline$x|(~(x-1))$\\
    this simply turns on/off the section
    left of the rightmost \lstinline$0$/\lstinline$1$
\item invert the original argument:\\
    \lstinline$(~x)&(x-1)$, \lstinline$(~x)|(x+1)$,
    \lstinline$(~x)&(x+1)$, \lstinline$(~x)&(x-1)$\\
\end{enumerate}

While the first two options deliver foreseeable results,
the inverting of the unchanged argument is the most complex
and promising option, as demonstrated in
\autoref{table:not-inc-and}, \autoref{table:not-dec-or},
\autoref{table:not-dec-and} and \autoref{table:not-inc-or}.

\begin{table}[H]
\centering
\begin{tabular}{r|ccc}
\lstinline$x$ & left of rightmost \lstinline$0$
    & rightmost \lstinline$0$ & trailing \lstinline$1$s\\
\hline
\lstinline$(~x)$ & inverted & \lstinline$1$ & \lstinline$0...0$\\
\lstinline$(x+1)$ & unchanged & \lstinline$1$ & \lstinline$0...0$\\
\lstinline$(~x)&(x+1)$ & \lstinline$0...0$
    & \lstinline$1$ & \lstinline$0...0$\\
& \multicolumn{3}{c}{
    \fbox{a single \lstinline$1$ at the rightmost \lstinline$0$}}
\end{tabular}
\caption{\lstinline$(NOT x) AND (INC x)$}
\label{table:not-inc-and}
\end{table}

\begin{table}[H]
\centering
\begin{tabular}{r|ccc}
\lstinline$x$ & left of rightmost \lstinline$1$
    & rightmost \lstinline$1$ & trailing \lstinline$0$s\\
\hline
\lstinline$(~x)$ & inverted & \lstinline$1$ & \lstinline$0...0$\\
\lstinline$(x-1)$ & unchanged & \lstinline$0$ & \lstinline$1...1$\\
\lstinline$(~x)|(x-1)$ & \lstinline$1...1$
    & \lstinline$0$ & \lstinline$1...1$\\
& \multicolumn{3}{c}{
    \fbox{a single \lstinline$0$ at the rightmost \lstinline$1$}}\\
\end{tabular}
\caption{\lstinline$(NOT x) OR (DEC x)$}
\label{table:not-dec-or}
\end{table}

\begin{table}[H]
\centering
\begin{tabular}{r|ccc}
\lstinline$x$ & left of rightmost \lstinline$1$
    & rightmost \lstinline$1$ & trailing \lstinline$0$s\\
\hline
\lstinline$(~x)$ & inverted & \lstinline$1$ & \lstinline$1...1$\\
\lstinline$(x-1)$ & unchanged & \lstinline$0$ & \lstinline$1...1$\\
\lstinline$(~x)&(x-1)$ & \lstinline$0...0$
    & \lstinline$0$ & \lstinline$1...1$\\
& \multicolumn{3}{c}{
    \fbox{\lstinline$1$s at trailing \lstinline$0$s}}\\
\end{tabular}
\caption{\lstinline$(NOT x) AND (DEC x)$}
\label{table:not-dec-and}
\end{table}

\begin{table}[H]
\centering
\begin{tabular}{r|ccc}
\lstinline$x$ & left of rightmost \lstinline$0$
    & rightmost \lstinline$0$ & trailing \lstinline$1$s\\
\hline
\lstinline$(~x)$ & inverted & \lstinline$1$ & \lstinline$1...1$\\
\lstinline$(x+1)$ & unchanged & \lstinline$1$ & \lstinline$0...0$\\
\lstinline$(~x)|(x+1)$ & \lstinline$1...1$
    & \lstinline$1$ & \lstinline$0...0$\\
& \multicolumn{3}{c}{
    \fbox{\lstinline$0$s at trailing \lstinline$1$s}}\\
\end{tabular}
\caption{\lstinline$(NOT x) OR (INC x)$}
\label{table:not-inc-or}
\end{table}

\subsubsection*{Inverting}
Additionally, every formula can be inverted in total.
This is especially useful when the result is ``a single \lstinline$0$''
like for \lstinline$(NOT x) OR (DEC x)$.
The result then has the same scheme as \lstinline$(NOT x) AND (INC x)$.
