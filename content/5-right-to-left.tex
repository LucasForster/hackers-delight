\section{Right to Left Computability Class}\label{sec:right-to-left}
After having developed
formulas modifying rightmost bits in \autoref{sec:combining},
the option arises to classify these kind of formulas.
\emph{Hacker's Delight} \cite{Warren:2012:HD:2462741}
therefore introduces the so called ``right to left computability class''.
Every formula in this class shares the property,
that each result bit only depends on input bits
at its position or to the right of it (see \autoref{table:property}).

\begin{table}[H]
\centering
\begin{tabular}{rrl}
\lstinline$x$: & \lstinline$x$$_n\ \ \dots$
    & \fbox{\lstinline$x$$_i$ \lstinline$x$$_{i-1} \dots$\lstinline$x$$_0$}\\
\lstinline$y$: & \lstinline$y$$_n\ \ \dots$
    & \fbox{\lstinline$y$$_i$ \lstinline$y$$_{i-1} \dots$\lstinline$y$$_0$}\\
result \lstinline$r$: & \lstinline$x$$_n\ \ \dots$
    & \fbox{\lstinline$r$$_i$}\lstinline$ r$$_{i-}\dots$\lstinline$r$$_{i-1}$\\
~\\
\multicolumn{3}{c}{\fbox{\lstinline$r$$_i$ only depends on
    \lstinline$x$$_i \dots$\lstinline$x$$_0$ and
    \lstinline$y$$_i \dots$\lstinline$y$$_0$}}\\
\end{tabular}
\caption{Right to left computability property}
\label{table:property}
\end{table}

Notice that, as it is common for a lot of operators,
the result bit \lstinline$r$$_i$ is allowed to depend on
\lstinline$r$$_{i-1}\dots$\lstinline$r$$_0$.
This is already given since
every of these result bits to the right of \lstinline$r$$_i$
only depend on bits of \lstinline$x$ and \lstinline$y$.

Before defining the class,
the previously used set of operators will get slightly modified:
\lstinline$INC$ and \lstinline$DEC$
get exchanged for \lstinline$ADD$ and \lstinline$SUB$.
This will allow for more flexibility while not using
more powerful operations: they can simply be expressed with loops.
The instruction set is therefore:
\[
    \text{\lstinline$NOT$},
    \underbrace{\text{\lstinline$ADD$, \lstinline$SUB$}}
    _{\text{\lstinline$INC$, \lstinline$DEC$}},
    \text{\lstinline$AND$},
    \text{\lstinline$OR$}
\]

It should also be mentioned, that this selection is a rational choice.
Smaller sets would also be possible,
and there are other right to left computable operators
-- they can be combined from the selected ones
(e.g.~shifting left as a sequence of \lstinline$ADD$s).

\begin{quote}
``THEOREM. A function mapping words to words can be implemented
with world-parallel \emph{add}, \emph{subtract}, \emph{or} and \emph{not}
instructions if and only if each bit of the result
depends only on bits at and to the right of each input operand.''
\par\hfill \emph{Hacker's Delight},
page \texttt{12} \cite{Warren:2012:HD:2462741}
\end{quote}

\begin{proof}~
\begin{description}
\item[``$\implies$'':]
In \autoref{sec:rightmost} \lstinline$INC$ and \lstinline$DEC$
where established as right to left operations.
This then also applies to \lstinline$ADD$ and \lstinline$SUB$.
Regarding \lstinline$NOT$, \lstinline$AND$, \lstinline$OR$,
they clearly fulfil the right to left computability property,
since their result bits do not on any adjacent ones at all.

Thereby a formula using only
\lstinline$ADD$, \lstinline$SUB$, \lstinline$AND$,
\lstinline$OR$ and \lstinline$NOT$
only depends on bits at and to the right of each input operand.

\item[``$\impliedby$'':]
The result only depends on bits at and to the right of each input operand.
These can be isolated using \lstinline$AND$ with the argument and a mask
(a constant containing a single \lstinline$1$ at the respective position).

Once every required bit is isolated, the function can be expressed
using only \lstinline$NOT$, \lstinline$OR$ and \lstinline$AND$.
This is true because (as mentioned in \autoref{sec:introduction})
$\{\lnot, \lor, \land\}$ is a complete set of logical operators,
every logical function can be expressed using only its elements.
\end{description}
\end{proof}

The obtained class contains all functions
that can calculate the $n$th result bit
with only knowing the input operands up to their $n$th bit.
This has applications e.g. in bitstreams when working on partial data.
